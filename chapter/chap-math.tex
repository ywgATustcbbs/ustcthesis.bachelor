
\chapter{数学公式}
\label{chap:math}
数学公式需在数学环境中排版。数学环境一般分\textit{正文公式}和\textit{显示公式}。正文公式用\verb|$...$|表示,而显示公式则用equation,align等数学环境。简单的几个例子:
\begin{figure}[ht]
\centering
\fbox{\begin{minipage}[h]{0.4\textwidth}
这是一个显示公式:
    \begin{equation}
    x+y=z
    \end{equation}
谢谢。
\end{minipage}}
\hspace{0.1\textwidth}
\begin{minipage}[h]{0.4\textwidth}
\centering
\begin{code}
这是一个显示公式:
\begin{equation}
x+y=z
\end{equation}
谢谢。
\end{code}
\end{minipage}
\caption{显示公式}
\end{figure}
\begin{figure}[h]
\centering
\fbox{\begin{minipage}[c]{0.4\textwidth}
大家好,$e^{\pi i}+1=0$是我最喜欢的方程。
\end{minipage}}
\hspace{0.1\textwidth}
\begin{minipage}[h]{0.4\textwidth}
\centering
\begin{code}
大家好,$e^{\pi i}+1=0$是我最喜欢的方程。
\end{code}
\end{minipage}
\caption{正文公式}
\end{figure}

常用的显示公式环境有:
\begin{itemize}
\item{equation}:最常用的公式环境。
\item{displaymath}:与equation的差别在于不会被自动编号。\\
\verb|\begin{displaymath}...\end{displaymath}|可以用\verb|\[...\]|来代替。
\item{align}:适用于排版多个需要对齐的公式。
\end{itemize}


\section{数学公式排版基础}
上下标、希腊字母、加减乘除等常用符号:
\begin{center}
\begin{tabular}{ccll}
\whline
& 命令 & 例子 & 代码\\
\hline
上标 & \^{} & $x^2$ & \verb|$x^2$|\\
下标 & \_{} & $x_2$ & \verb|$x_2$|\\
希腊字母 & 对应的英文 & $\pi,\rho,\sigma,...$ &\verb|$\pi,\rho,\sigma,...$|\\
点乘 & \verb|\cdot| & $x\cdot y$ & \verb|$x\cdot y$|\\
叉乘 & \verb|\times| & $x\times y$ & \verb|$x\times y$|\\
分数 & \verb|\frac{}{}| & $\frac{abc}{def}$ &\verb|$\frac{abc}{def}$|\\
根号 & \verb|\sqrt[n]{}|  & $\sqrt{x},\sqrt[3]{y}$ &\verb|$\sqrt{x},\sqrt[3]{y}$|\\
积分 &\verb|\int|  & $\int,\iint,\iiint$ & \verb|$\int,\iint,\iiint$|\\
\whline
\end{tabular}
\end{center}
表中,\$代表数学环境。在equation等环境中排版时,不再需要使用\$符号。

更多的符号,请点击WinEdt面板上的$\Sigma$符号(TeX Symbols GUI),在弹出的面板中查找。

当控制符作用于多个字符时,需要用大括号将这些字符都括起来,如
\begin{center}
\begin{tabular}{c c}
\whline
例子 & 代码\\
\hline
$e^{\pi i}+1=0$ & \verb|e^{\pi i}+1=0|\\
$x_{12}^{3}$ & \verb|x_{12}^{3}|\\
${x_{12}}^{3}$ & \verb|{x_{12}}^{3}|\\
\whline
\end{tabular}
\end{center}

\section{控制字体}
本科学位论文对于公式内字体的使用有严格的规定。一般变量用白斜体,常量和特殊函数等用白正体,矢量用黑斜体,张量用黑正体。改变字体的命令如下表:
\begin{center}
\begin{tabular}{c c l l}
\whline
& 字体 & 例子 & 代码\\
\hline
一般变量 & 白斜体 & $x$ & \verb|$x$|\\
常量  & 白正体 & $\mathrm{c}$  & \verb|$\mathrm{c}$|\\
矢量  & 黑斜体 & $\bm{B}$ & \verb|$\bm{B}$|\\
张量  & 黑正体 & $\mathbf{H}$ & \verb|$\mathbf{H}$|\\
\whline
\end{tabular}
\end{center}
表中,\$代表数学环境。在equation等环境中排版时,不再需要使用\$符号。

\section{括号}
中括号、小括号和竖线可以用键盘上对应的键来输入。大括号是\LaTeX{}中的控制符,必须用\verb|\{...\}|来输入。态函数所用的尖括号是\verb|\langle|和\verb|\rangle|。一个句点代表一个不显示的括号。

使用\verb|\left|和\verb|\right|可以自动控制括号的大小。使用时,将所需要用的括号紧跟在\verb|\left|或\verb|\right|后面就好。例如:
\begin{figure}[ht]
\centering
\fbox{\begin{minipage}[h]{0.8\textwidth}
    \[
    \left[\frac{\left(\frac{abc}{def}+ghi\right)\cdot jkl}{mno}+pqr\right]\times st=0
    \]
\end{minipage}}\\
\vskip10pt
\begin{minipage}[h]{0.8\textwidth}
\centering
\begin{code}
\[
\left[\frac{\left(\frac{abc}{def}+ghi\right)\cdot jkl}{mno}+pqr\right]
\times st=0
\]
\end{code}
\end{minipage}
\caption{括号。\textbackslash[和\textbackslash]代表displaymath环境,下同。}
\end{figure}



使用括号时,不一定要用相同的括号来匹配。甚至可以用\verb|\left.|这样不被显示括号来匹配正常括号,以达到排版目的。如
\begin{figure}[h]
\centering
\fbox{\begin{minipage}[h]{0.4\textwidth}
\[\left|\psi\right\rangle\]
\end{minipage}}
\hspace{0.1\textwidth}
\begin{minipage}[h]{0.4\textwidth}
\centering
\begin{code}
\[\left|\psi\right\rangle\]
\end{code}
\end{minipage}\\
\vskip10pt
\fbox{\begin{minipage}[h]{0.4\textwidth}
\[\left\{
\begin{array}{c}
a\\b\\c
\end{array}
\right.\]
\end{minipage}}
\hspace{0.1\textwidth}
\begin{minipage}[h]{0.4\textwidth}
\centering
\begin{code}
\[\left\{\begin{array}{c}
a\\b\\c
\end{array}\right.\]
\end{code}
\end{minipage}
\caption{不同括号的匹配}
\end{figure}

\section{矩阵}
输入矩阵一般使用array环境。

矩阵中使用\verb|&|来分位,\verb|\\|来分行。array环境的参数指定了列的数量和每列的对齐方式。c代表一个居中对齐的列,l是靠左对齐,r则是靠右对齐。图\ref{f:matrix}是一个简单的例子。
\begin{figure}[h]
\centering
\fbox{\begin{minipage}[h]{0.4\textwidth}
\[\left[
\begin{array}{c c c}
C_{Xx} & C_{Yx} & C_{Zx}\\
C_{Xy} & C_{Yy} & C_{Zy}\\
C_{Xz} & C_{Yz} & C_{Zz}
\end{array}
\right]\]
\end{minipage}}
\hspace{0.1\textwidth}
\begin{minipage}[h]{0.4\textwidth}
\centering
\begin{code}
\[\left[\begin{array}{c c c}
C_{Xx} & C_{Yx} & C_{Zx}\\
C_{Xy} & C_{Yy} & C_{Zy}\\
C_{Xz} & C_{Yz} & C_{Zz}
\end{array}\right]\]
\end{code}
\end{minipage}
\caption{矩阵}
\label{f:matrix}
\end{figure}

array环境可以自嵌套,达到输入复杂矩阵的目的。

\section{对齐的多行公式}
align环境提供了书写多行公式,并指定其在某一位置对齐的功能,使用\&来代表对齐点。如图\ref{f:align}。其中,\verb|\notag| 表示不对这一行公式进行编号。
\begin{figure}[h]
\centering
\fbox{\begin{minipage}[b]{0.8\textwidth}
\begin{align}
a_0&=(A_1+A_2+A_3)/3\textrm{g}_e\beta_e\\
b_0&=[A_1-(A_2+A_3)/2]/3\textrm{g}_e\beta_e\notag\\
c_0&=(|A_2|-|A_3|)/2\textrm{g}_e\beta_e
\end{align}
\end{minipage}}\\
\begin{minipage}[b]{0.8\textwidth}
\centering
\begin{code}
\begin{align}
a_0&=(A_1+A_2+A_3)/3\textrm{g}_e\beta_e\\
b_0&=[A_1-(A_2+A_3)/2]/3\textrm{g}_e\beta_e\notag\\
c_0&=(|A_2|-|A_3|)/2\textrm{g}_e\beta_e
\end{align}
\end{code}
\end{minipage}
\caption{多行公式}
\label{f:align}
\end{figure}

\section{定理,证明,和其他环境}
ustcthesis中提供了定理,证明等一些其他环境。环境和名称如下表:
\begin{center}
\begin{tabular}{l l @\qquad l l}
\whline
环境名 & 中文名 &环境名 & 中文名\\
\hline
theorem & 定理&
lemma&引理\\
example&例&
algorithm&算法\\
definition&定义&
axiom&公理\\
property&性质&
proposition&命题\\
corollary&推论&
remark&注解\\
condition&条件&
conclusion&结论\\
assumption&假设&
prove&证明\\
\whline
\end{tabular}
\end{center}
